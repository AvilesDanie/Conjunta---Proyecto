\documentclass[12pt,a4paper]{article}
\usepackage[utf8]{inputenc}
\usepackage[spanish]{babel}
\usepackage{geometry}
\geometry{a4paper, margin=2.5cm}
\usepackage{graphicx}
\usepackage{longtable}
\usepackage{array}
\usepackage{booktabs}
\usepackage{xcolor}
\usepackage{fancyhdr}
\usepackage{enumitem}
\usepackage{amsmath}

% Configuración de encabezado y pie de página
\pagestyle{fancy}
\fancyhf{}
\lhead{Banco BanQuito - Módulo de Crédito}
\rhead{Item 2 - Descripción de Caso de Uso}
\cfoot{\thepage}

% Colores personalizados
\definecolor{headercolor}{RGB}{0,102,204}
\definecolor{lightgray}{RGB}{240,240,240}

\title{\textbf{DESCRIPCIÓN DEL CASO DE USO\\[0.5cm]
\Large Obtener Monto Máximo de Crédito}}
\author{Sistema Banco BanQuito - Módulo de Crédito\\
Examen Complexivo - Arquitectura de Software}
\date{Octubre 2025}

\begin{document}

\maketitle
\thispagestyle{empty}

\vspace{1cm}

\begin{center}
\begin{tabular}{|p{5cm}|p{9cm}|}
\hline
\textbf{ID del Caso de Uso} & CU-02 \\
\hline
\textbf{Nombre} & Obtener Monto Máximo de Crédito \\
\hline
\textbf{Actor Principal} & Sistema Comercializadora de Electrodomésticos \\
\hline
\textbf{Tipo} & Servicio Web SOAP \\
\hline
\textbf{Prioridad} & Alta \\
\hline
\textbf{Puntos Rúbrica} & Item 6 - 1.5 puntos \\
\hline
\end{tabular}
\end{center}

\newpage

\tableofcontents

\newpage

\section{Descripción General}

Este caso de uso permite calcular el monto máximo de crédito que puede ser otorgado a un cliente del Banco BanQuito, basándose en el análisis de sus movimientos bancarios de los últimos 3 meses. El sistema evaluará los promedios de depósitos y retiros para determinar la capacidad crediticia del cliente.

\subsection{Objetivo}

Determinar de forma automática el límite de crédito que puede ser aprobado para un cliente, aplicando una fórmula financiera que considera sus ingresos (depósitos) y gastos (retiros) históricos.

\section{Precondiciones}

\begin{enumerate}[label=\arabic*.]
    \item El cliente debe existir en la base de datos del sistema CORE (tabla CLIENTE)
    \item El cliente debe estar registrado con una cédula válida (10 dígitos)
    \item El cliente debe tener al menos una cuenta asociada (tabla CUENTA)
    \item Debe existir historial de movimientos en los últimos 3 meses (tabla MOVIMIENTO)
    \item \textit{Recomendado:} El cliente debe haber pasado la validación de "Sujeto de Crédito" (CU-01)
\end{enumerate}

\section{Postcondiciones}

\subsection{Postcondición de Éxito}
\begin{itemize}
    \item El sistema retorna el monto máximo de crédito calculado como un valor decimal (ejemplo: 3456.78)
    \item El monto es expresado en dólares estadounidenses (USD)
    \item El monto es mayor a 0 (cero)
\end{itemize}

\subsection{Postcondición de Fallo}
\begin{itemize}
    \item El sistema retorna monto \$0.00 si no hay suficiente historial de movimientos
    \item El sistema retorna monto \$0.00 si la diferencia entre promedios es negativa o cero
    \item El sistema retorna un mensaje de error SOAP si la cédula no existe
\end{itemize}

\newpage

\section{Flujo Principal (Flujo Básico)}

\subsection{Acciones del Actor}

\begin{enumerate}[label=\textbf{\arabic*.}, leftmargin=1.5cm]
    \item El Sistema Comercializadora invoca el servicio web \texttt{obtenerMontoMaximoCredito(cedula)} con la cédula del cliente como parámetro.
    
    \item El Sistema Comercializadora recibe el monto máximo aprobado.
\end{enumerate}

\subsection{Respuestas del Sistema}

\begin{enumerate}[label=\textbf{\arabic*.}, leftmargin=1.5cm]
    \setcounter{enumi}{1}
    
    \item El sistema valida que la cédula tenga formato correcto (10 dígitos numéricos).
    
    \item El sistema busca al cliente en la tabla CLIENTE usando la cédula proporcionada.
    
    \item El sistema verifica que el cliente existe. Si no existe, ejecuta \textbf{Flujo Alternativo 1}.
    
    \item El sistema obtiene el número de cuenta(s) asociada(s) al cliente desde la tabla CUENTA.
    
    \item El sistema calcula la fecha límite: \textbf{Fecha actual - 90 días}.
    
    \item \textbf{[Sub-proceso: Calcular Promedio de Depósitos]} 
    
    El sistema ejecuta la consulta SQL:
    \begin{verbatim}
    SELECT AVG(VALOR) 
    FROM MOVIMIENTO 
    WHERE NUM_CUENTA IN (cuentas_cliente) 
      AND TIPO = 'D' 
      AND FECHA >= fecha_limite
    \end{verbatim}
    
    \item El sistema almacena el resultado como \textbf{PromedioDepositos}.
    
    \item \textbf{[Sub-proceso: Calcular Promedio de Retiros]}
    
    El sistema ejecuta la consulta SQL:
    \begin{verbatim}
    SELECT AVG(VALOR) 
    FROM MOVIMIENTO 
    WHERE NUM_CUENTA IN (cuentas_cliente) 
      AND TIPO = 'R' 
      AND FECHA >= fecha_limite
    \end{verbatim}
    
    \item El sistema almacena el resultado como \textbf{PromedioRetiros}.
    
    \item El sistema verifica que ambos promedios sean diferentes de NULL. Si alguno es NULL, ejecuta \textbf{Flujo Alternativo 2}.
    
    \item \textbf{[Sub-proceso: Aplicar Fórmula]} 
    
    El sistema calcula:
    \begin{equation}
    MontoMaximo = ((PromedioDepositos - PromedioRetiros) \times 0.60) \times 9
    \end{equation}
    
    \item El sistema verifica que $MontoMaximo > 0$. Si es $\leq 0$, ejecuta \textbf{Flujo Alternativo 3}.
    
    \item El sistema redondea el MontoMaximo a 2 decimales.
    
    \item El sistema retorna el MontoMaximo en formato decimal al servicio web.
\end{enumerate}

\newpage

\section{Flujos Alternativos}

\subsection{Flujo Alternativo 1: Cliente no existe}

\begin{itemize}
    \item \textbf{Paso donde se bifurca:} Paso 4
    \item \textbf{Condición:} La cédula no se encuentra en la tabla CLIENTE
    \item \textbf{Acciones:}
    \begin{enumerate}
        \item El sistema registra en el log: "Cliente con cédula [X] no encontrado"
        \item El sistema retorna una excepción SOAP con el mensaje: "ERROR: Cliente no registrado en el banco"
        \item El flujo termina
    \end{enumerate}
\end{itemize}

\subsection{Flujo Alternativo 2: Sin historial de movimientos}

\begin{itemize}
    \item \textbf{Paso donde se bifurca:} Paso 11
    \item \textbf{Condición:} No existen depósitos o retiros en los últimos 3 meses
    \item \textbf{Acciones:}
    \begin{enumerate}
        \item El sistema asigna \$0.00 a los promedios que sean NULL
        \item El sistema continúa al paso 12
        \item Como resultado, MontoMaximo será 0 o negativo, activando el Flujo Alternativo 3
    \end{enumerate}
\end{itemize}

\subsection{Flujo Alternativo 3: Monto calculado es cero o negativo}

\begin{itemize}
    \item \textbf{Paso donde se bifurca:} Paso 13
    \item \textbf{Condición:} $(PromedioDepositos - PromedioRetiros) \leq 0$ o resultado final $\leq 0$
    \item \textbf{Acciones:}
    \begin{enumerate}
        \item El sistema asigna $MontoMaximo = \$0.00$
        \item El sistema retorna \$0.00 como monto máximo
        \item El flujo termina (indica que el cliente no tiene capacidad de pago)
    \end{enumerate}
\end{itemize}

\newpage

\section{Flujos de Excepción}

\subsection{Excepción 1: Error de conexión a la base de datos}
\begin{itemize}
    \item \textbf{Condición:} Falla la conexión JDBC en cualquier consulta SQL
    \item \textbf{Acción:} 
    \begin{itemize}
        \item Sistema retorna SOAP Fault con código "DB\_CONNECTION\_ERROR"
        \item Mensaje: "Error al conectar con la base de datos. Intente nuevamente."
        \item Registra error en log del servidor
    \end{itemize}
\end{itemize}

\subsection{Excepción 2: Cédula con formato inválido}
\begin{itemize}
    \item \textbf{Condición:} Cédula no es numérica o no tiene 10 dígitos
    \item \textbf{Acción:}
    \begin{itemize}
        \item Sistema retorna SOAP Fault con código "INVALID\_CEDULA\_FORMAT"
        \item Mensaje: "Formato de cédula inválido. Debe contener 10 dígitos numéricos."
    \end{itemize}
\end{itemize}

\subsection{Excepción 3: Error en cálculo matemático}
\begin{itemize}
    \item \textbf{Condición:} División por cero o valor NULL inesperado en cálculo
    \item \textbf{Acción:}
    \begin{itemize}
        \item Sistema registra el error en log
        \item Sistema retorna $MontoMaximo = \$0.00$
        \item Continúa flujo (no interrumpe)
    \end{itemize}
\end{itemize}

\newpage

\section{Reglas de Negocio}

\begin{longtable}{|c|p{4cm}|p{8cm}|}
\hline
\textbf{ID} & \textbf{Regla} & \textbf{Descripción} \\
\hline
\endfirsthead

\hline
\textbf{ID} & \textbf{Regla} & \textbf{Descripción} \\
\hline
\endhead

RN-01 & Ventana de cálculo & Se consideran únicamente los últimos 90 días (3 meses) desde la fecha actual \\
\hline

RN-02 & Tipos de movimiento & TIPO = 'D' para depósitos, TIPO = 'R' para retiros \\
\hline

RN-03 & Fórmula financiera & $MontoMax = ((AvgDep - AvgRet) \times 60\%) \times 9$ \newline Factor 0.60 = 60\% de capacidad de pago \newline Factor 9 = Multiplicador de 9 meses \\
\hline

RN-04 & Múltiples cuentas & Si el cliente tiene varias cuentas, se suman todos los movimientos de todas las cuentas \\
\hline

RN-05 & Monto mínimo & Si el resultado es menor o igual a 0, se retorna \$0.00 (no créditos negativos) \\
\hline

RN-06 & Precisión decimal & Resultado redondeado a 2 decimales (centavos) \\
\hline

\caption{Reglas de Negocio del Caso de Uso}
\end{longtable}

\section{Datos de Entrada y Salida}

\subsection{Parámetros de Entrada}

\begin{center}
\begin{tabular}{|l|l|c|p{5cm}|l|}
\hline
\textbf{Parámetro} & \textbf{Tipo} & \textbf{Obligatorio} & \textbf{Descripción} & \textbf{Ejemplo} \\
\hline
cedula & String & Sí & Cédula de identidad del cliente (10 dígitos) & "1712345678" \\
\hline
\end{tabular}
\end{center}

\subsection{Datos de Salida}

\begin{center}
\begin{tabular}{|l|l|p{6cm}|l|}
\hline
\textbf{Campo} & \textbf{Tipo} & \textbf{Descripción} & \textbf{Ejemplo} \\
\hline
montoMaximo & Decimal(10,2) & Monto máximo de crédito aprobado en USD & 3456.78 \\
\hline
\end{tabular}
\end{center}

\newpage

\section{Ejemplo de Cálculo Completo}

\subsection{Datos de Entrada}
\begin{itemize}
    \item Cédula: "1712345678"
    \item Fecha actual: 2025-10-29
    \item Fecha límite: 2025-07-31 (90 días atrás)
\end{itemize}

\subsection{Movimientos en Base de Datos (últimos 3 meses)}

\textbf{Depósitos (TIPO='D'):}
\begin{itemize}
    \item 2025-08-15: \$1200.00
    \item 2025-09-10: \$1500.00
    \item 2025-10-05: \$1300.00
\end{itemize}

\textbf{Retiros (TIPO='R'):}
\begin{itemize}
    \item 2025-08-20: \$400.00
    \item 2025-09-15: \$350.00
    \item 2025-10-10: \$450.00
\end{itemize}

\subsection{Proceso de Cálculo}

\begin{align*}
\text{PromedioDepositos} &= \frac{1200 + 1500 + 1300}{3} = \$1333.33 \\[0.3cm]
\text{PromedioRetiros} &= \frac{400 + 350 + 450}{3} = \$400.00 \\[0.3cm]
\text{Diferencia} &= 1333.33 - 400.00 = \$933.33 \\[0.3cm]
\text{60\% de diferencia} &= 933.33 \times 0.60 = \$560.00 \\[0.3cm]
\text{MontoMaximo} &= 560.00 \times 9 = \$5040.00
\end{align*}

\subsection{Resultado Final}
\begin{center}
\fcolorbox{headercolor}{lightgray}{
\parbox{0.8\textwidth}{
\centering
\textbf{\Large Monto Máximo de Crédito Aprobado:}\\[0.3cm]
\textbf{\Huge \$5,040.00}
}
}
\end{center}

\newpage

\section{Requisitos Especiales}

\subsection{Requisitos No Funcionales}

\begin{enumerate}
    \item \textbf{Rendimiento:} El servicio debe responder en menos de 3 segundos
    \item \textbf{Disponibilidad:} Servicio disponible 24/7 (99.5\% uptime)
    \item \textbf{Seguridad:}
    \begin{itemize}
        \item Validación de formato de cédula antes de consultar BD
        \item Protección contra SQL Injection (uso de PreparedStatement)
        \item Cifrado de comunicación SOAP (HTTPS recomendado)
    \end{itemize}
    \item \textbf{Escalabilidad:} Soportar hasta 100 consultas concurrentes
    \item \textbf{Logging:} Registrar cada invocación con: timestamp, cédula consultada, resultado
\end{enumerate}

\subsection{Requisitos de Datos}

\begin{enumerate}
    \item Cédula debe ser string de 10 caracteres numéricos
    \item Montos deben usar tipo DECIMAL(10,2) para evitar pérdida de precisión
    \item Fechas en formato ISO 8601 (yyyy-MM-dd)
    \item Cálculos financieros con precisión de 2 decimales
\end{enumerate}

\section{Casos de Prueba}

\begin{longtable}{|c|l|p{5cm}|p{4cm}|}
\hline
\textbf{ID} & \textbf{Cédula} & \textbf{Escenario} & \textbf{Resultado Esperado} \\
\hline
\endfirsthead

\hline
\textbf{ID} & \textbf{Cédula} & \textbf{Escenario} & \textbf{Resultado Esperado} \\
\hline
\endhead

TC-01 & 1712345678 & Cliente con movimientos normales & Monto > 0 (ej: \$5040.00) \\
\hline

TC-02 & 1798765432 & Cliente sin movimientos últimos 3 meses & Monto = \$0.00 \\
\hline

TC-03 & 1700000001 & Cliente con más retiros que depósitos & Monto = \$0.00 \\
\hline

TC-04 & 9999999999 & Cliente no existe en BD & SOAP Fault: "Cliente no registrado" \\
\hline

TC-05 & ABC123XYZ0 & Formato de cédula inválido & SOAP Fault: "Formato inválido" \\
\hline

TC-06 & 1712345678 & Cliente con múltiples cuentas & Suma movimientos de todas las cuentas \\
\hline

\caption{Casos de Prueba del Caso de Uso}
\end{longtable}

\newpage

\section{Matriz de Trazabilidad}

\begin{center}
\begin{tabular}{|p{7cm}|c|c|}
\hline
\textbf{Requisito Funcional} & \textbf{Item Rúbrica} & \textbf{Prioridad} \\
\hline
Consultar movimientos de últimos 3 meses & Item 6 & Alta \\
\hline
Calcular promedio de depósitos & Item 6 & Alta \\
\hline
Calcular promedio de retiros & Item 6 & Alta \\
\hline
Aplicar fórmula: ((AvgDep-AvgRet)*60\%)*9 & Item 6 & Alta \\
\hline
Retornar monto vía Web Service SOAP & Item 6 & Alta \\
\hline
\end{tabular}
\end{center}

\section{Ejemplo de Mensaje SOAP}

\subsection{Petición SOAP (Request)}

\begin{verbatim}
<?xml version="1.0" encoding="UTF-8"?>
<soap:Envelope xmlns:soap="http://schemas.xmlsoap.org/soap/envelope/"
               xmlns:ban="http://banquito.com/credito">
  <soap:Body>
    <ban:ObtenerMontoMaximoRequest>
      <cedula>1712345678</cedula>
    </ban:ObtenerMontoMaximoRequest>
  </soap:Body>
</soap:Envelope>
\end{verbatim}

\subsection{Respuesta SOAP (Response - Éxito)}

\begin{verbatim}
<?xml version="1.0" encoding="UTF-8"?>
<soap:Envelope xmlns:soap="http://schemas.xmlsoap.org/soap/envelope/"
               xmlns:ban="http://banquito.com/credito">
  <soap:Body>
    <ban:ObtenerMontoMaximoResponse>
      <montoMaximo>5040.00</montoMaximo>
    </ban:ObtenerMontoMaximoResponse>
  </soap:Body>
</soap:Envelope>
\end{verbatim}

\subsection{Respuesta SOAP (Fault - Error)}

\begin{verbatim}
<?xml version="1.0" encoding="UTF-8"?>
<soap:Envelope xmlns:soap="http://schemas.xmlsoap.org/soap/envelope/">
  <soap:Body>
    <soap:Fault>
      <faultcode>INVALID_CEDULA_FORMAT</faultcode>
      <faultstring>Formato de cédula inválido. 
                   Debe contener 10 dígitos numéricos.</faultstring>
    </soap:Fault>
  </soap:Body>
</soap:Envelope>
\end{verbatim}

\end{document}
